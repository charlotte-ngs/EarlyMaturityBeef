\documentclass[]{article}
\usepackage{lmodern}
\usepackage{amssymb,amsmath}
\usepackage{ifxetex,ifluatex}
\usepackage{fixltx2e} % provides \textsubscript
\ifnum 0\ifxetex 1\fi\ifluatex 1\fi=0 % if pdftex
  \usepackage[T1]{fontenc}
  \usepackage[utf8]{inputenc}
\else % if luatex or xelatex
  \ifxetex
    \usepackage{mathspec}
  \else
    \usepackage{fontspec}
  \fi
  \defaultfontfeatures{Ligatures=TeX,Scale=MatchLowercase}
\fi
% use upquote if available, for straight quotes in verbatim environments
\IfFileExists{upquote.sty}{\usepackage{upquote}}{}
% use microtype if available
\IfFileExists{microtype.sty}{%
\usepackage{microtype}
\UseMicrotypeSet[protrusion]{basicmath} % disable protrusion for tt fonts
}{}
\usepackage[margin=1in]{geometry}
\usepackage{hyperref}
\hypersetup{unicode=true,
            pdftitle={Response to selection},
            pdfauthor={Silvan},
            pdfborder={0 0 0},
            breaklinks=true}
\urlstyle{same}  % don't use monospace font for urls
\usepackage{color}
\usepackage{fancyvrb}
\newcommand{\VerbBar}{|}
\newcommand{\VERB}{\Verb[commandchars=\\\{\}]}
\DefineVerbatimEnvironment{Highlighting}{Verbatim}{commandchars=\\\{\}}
% Add ',fontsize=\small' for more characters per line
\usepackage{framed}
\definecolor{shadecolor}{RGB}{248,248,248}
\newenvironment{Shaded}{\begin{snugshade}}{\end{snugshade}}
\newcommand{\KeywordTok}[1]{\textcolor[rgb]{0.13,0.29,0.53}{\textbf{#1}}}
\newcommand{\DataTypeTok}[1]{\textcolor[rgb]{0.13,0.29,0.53}{#1}}
\newcommand{\DecValTok}[1]{\textcolor[rgb]{0.00,0.00,0.81}{#1}}
\newcommand{\BaseNTok}[1]{\textcolor[rgb]{0.00,0.00,0.81}{#1}}
\newcommand{\FloatTok}[1]{\textcolor[rgb]{0.00,0.00,0.81}{#1}}
\newcommand{\ConstantTok}[1]{\textcolor[rgb]{0.00,0.00,0.00}{#1}}
\newcommand{\CharTok}[1]{\textcolor[rgb]{0.31,0.60,0.02}{#1}}
\newcommand{\SpecialCharTok}[1]{\textcolor[rgb]{0.00,0.00,0.00}{#1}}
\newcommand{\StringTok}[1]{\textcolor[rgb]{0.31,0.60,0.02}{#1}}
\newcommand{\VerbatimStringTok}[1]{\textcolor[rgb]{0.31,0.60,0.02}{#1}}
\newcommand{\SpecialStringTok}[1]{\textcolor[rgb]{0.31,0.60,0.02}{#1}}
\newcommand{\ImportTok}[1]{#1}
\newcommand{\CommentTok}[1]{\textcolor[rgb]{0.56,0.35,0.01}{\textit{#1}}}
\newcommand{\DocumentationTok}[1]{\textcolor[rgb]{0.56,0.35,0.01}{\textbf{\textit{#1}}}}
\newcommand{\AnnotationTok}[1]{\textcolor[rgb]{0.56,0.35,0.01}{\textbf{\textit{#1}}}}
\newcommand{\CommentVarTok}[1]{\textcolor[rgb]{0.56,0.35,0.01}{\textbf{\textit{#1}}}}
\newcommand{\OtherTok}[1]{\textcolor[rgb]{0.56,0.35,0.01}{#1}}
\newcommand{\FunctionTok}[1]{\textcolor[rgb]{0.00,0.00,0.00}{#1}}
\newcommand{\VariableTok}[1]{\textcolor[rgb]{0.00,0.00,0.00}{#1}}
\newcommand{\ControlFlowTok}[1]{\textcolor[rgb]{0.13,0.29,0.53}{\textbf{#1}}}
\newcommand{\OperatorTok}[1]{\textcolor[rgb]{0.81,0.36,0.00}{\textbf{#1}}}
\newcommand{\BuiltInTok}[1]{#1}
\newcommand{\ExtensionTok}[1]{#1}
\newcommand{\PreprocessorTok}[1]{\textcolor[rgb]{0.56,0.35,0.01}{\textit{#1}}}
\newcommand{\AttributeTok}[1]{\textcolor[rgb]{0.77,0.63,0.00}{#1}}
\newcommand{\RegionMarkerTok}[1]{#1}
\newcommand{\InformationTok}[1]{\textcolor[rgb]{0.56,0.35,0.01}{\textbf{\textit{#1}}}}
\newcommand{\WarningTok}[1]{\textcolor[rgb]{0.56,0.35,0.01}{\textbf{\textit{#1}}}}
\newcommand{\AlertTok}[1]{\textcolor[rgb]{0.94,0.16,0.16}{#1}}
\newcommand{\ErrorTok}[1]{\textcolor[rgb]{0.64,0.00,0.00}{\textbf{#1}}}
\newcommand{\NormalTok}[1]{#1}
\usepackage{longtable,booktabs}
\usepackage{graphicx,grffile}
\makeatletter
\def\maxwidth{\ifdim\Gin@nat@width>\linewidth\linewidth\else\Gin@nat@width\fi}
\def\maxheight{\ifdim\Gin@nat@height>\textheight\textheight\else\Gin@nat@height\fi}
\makeatother
% Scale images if necessary, so that they will not overflow the page
% margins by default, and it is still possible to overwrite the defaults
% using explicit options in \includegraphics[width, height, ...]{}
\setkeys{Gin}{width=\maxwidth,height=\maxheight,keepaspectratio}
\IfFileExists{parskip.sty}{%
\usepackage{parskip}
}{% else
\setlength{\parindent}{0pt}
\setlength{\parskip}{6pt plus 2pt minus 1pt}
}
\setlength{\emergencystretch}{3em}  % prevent overfull lines
\providecommand{\tightlist}{%
  \setlength{\itemsep}{0pt}\setlength{\parskip}{0pt}}
\setcounter{secnumdepth}{0}
% Redefines (sub)paragraphs to behave more like sections
\ifx\paragraph\undefined\else
\let\oldparagraph\paragraph
\renewcommand{\paragraph}[1]{\oldparagraph{#1}\mbox{}}
\fi
\ifx\subparagraph\undefined\else
\let\oldsubparagraph\subparagraph
\renewcommand{\subparagraph}[1]{\oldsubparagraph{#1}\mbox{}}
\fi

%%% Use protect on footnotes to avoid problems with footnotes in titles
\let\rmarkdownfootnote\footnote%
\def\footnote{\protect\rmarkdownfootnote}

%%% Change title format to be more compact
\usepackage{titling}

% Create subtitle command for use in maketitle
\newcommand{\subtitle}[1]{
  \posttitle{
    \begin{center}\large#1\end{center}
    }
}

\setlength{\droptitle}{-2em}
  \title{Response to selection}
  \pretitle{\vspace{\droptitle}\centering\huge}
  \posttitle{\par}
  \author{Silvan}
  \preauthor{\centering\large\emph}
  \postauthor{\par}
  \predate{\centering\large\emph}
  \postdate{\par}
  \date{20 8 2018}


\begin{document}
\maketitle

The change produced by selection that chiefly interests us is the change
of the population mean. Th is is the response to selection, which will
be symbolized by R. It is the difference of mean phenotypic value
between the offspring of the selcted parents and the whole of the
parental generation before selection. The measure of the selection
applied is the average superiority of the selected parents, which is
called the selection differential, and will be symbolized by S. If you
can assume that there is no natural selection (fertility and viability
not correlated with trait), you can say the ratio of response to
selection differential is equal to the heritability (Falconer and Mackay
1996). \[R = h^2S\] Assortative mating can be disregarded, because it
has very little effect on the mid-parent regression. The chief use of
this equation is for predicting the response to selection. The knowledge
of heritability can be obtained from previous generations. The selection
differential can not be known, but its expected value can be predicted.

In the following example you can see how the response to selection for
abdominal bristle number in drosophila.

\begin{verbatim}
## Warning: package 'knitr' was built under R version 3.4.3
\end{verbatim}

\begin{longtable}[]{@{}lrllll@{}}
\toprule
Generation & Mean of all measured & Mean of those selected & Selection
differential & Experienced response & Observed response\tabularnewline
\midrule
\endhead
Parents & 35.3 & 40.6 & 5.3 & 2.8 & -\tabularnewline
Offspring & 37.9 & - & - & - & 2.6\tabularnewline
\bottomrule
\end{longtable}

\begin{Shaded}
\begin{Highlighting}[]
\NormalTok{h2 <-}\StringTok{ }\FloatTok{0.52}
\NormalTok{S <-}\StringTok{ }\FloatTok{40.6} \OperatorTok{-}\StringTok{ }\FloatTok{35.3}
\NormalTok{R <-}\StringTok{ }\NormalTok{h2 }\OperatorTok{*}\StringTok{ }\NormalTok{S}
\NormalTok{oR <-}\StringTok{ }\FloatTok{37.9} \OperatorTok{-}\StringTok{ }\FloatTok{35.3}
\end{Highlighting}
\end{Shaded}

The heritability 0.52 of the bristle number was estimated in the
previous generations. The selected parents had a mean superiority of
5.3. The predicted response is S * h2 = 2.756. The observed response was
2.6. However this result is only valid for this generation, because
heritability is expected to change in each generation.

When a normal distribution of the values for a trait in a population can
be assumed, which is usually the case you can derivate the intensity of
selection of the proportion selected.

The goal is of course to increase the response to selection when we are
breeding. Increasing intensity of selection seems at first sight to be a
straightforward way of improving the response. However there are two
factors that limit increasing intensity of selection: 1. The
reproductive rate 2. Inbreeding

The response to selection is dependent on the parameters:

\begin{enumerate}
\def\labelenumi{\arabic{enumi}.}
\tightlist
\item
  Selection intensity (si)
\item
  Selection accuracy (sa)
\item
  Genetic standard deviation (gsd)
\item
  Generation intervall {[}y{]} (gi)
\item
  The sex of the candidate (m/f)
\end{enumerate}

In this example I calculate the response to selection for the both sexes
as shown on page 135 (Von Rohr et al. 1998). The abbreviation ``m or f''
in front of the parameters will show if the parameter is from a male or
female candidate. The first R chunk will define the necessary paramters:

\begin{Shaded}
\begin{Highlighting}[]
\NormalTok{msi <-}\StringTok{ }\FloatTok{1.4}
\NormalTok{mgi <-}\StringTok{ }\FloatTok{1.25}
\NormalTok{fsi <-}\StringTok{ }\FloatTok{0.8}
\NormalTok{fgi <-}\StringTok{ }\FloatTok{1.75}
\NormalTok{msa <-}\StringTok{ }\DecValTok{500}
\NormalTok{gsd <-}\StringTok{ }\DecValTok{500}
\NormalTok{fsa <-}\StringTok{ }\DecValTok{500}
\end{Highlighting}
\end{Shaded}

The second R chunk will calculate the asymptotic response to selection
(R) per year for the total breeding value:

\begin{Shaded}
\begin{Highlighting}[]
\NormalTok{x <-}\StringTok{ }\NormalTok{msi }\OperatorTok{*}\StringTok{ }\NormalTok{msa }\OperatorTok{*}\StringTok{ }\NormalTok{gsd }\OperatorTok{+}\StringTok{ }\NormalTok{fsi }\OperatorTok{*}\StringTok{ }\NormalTok{fsa }\OperatorTok{*}\StringTok{ }\NormalTok{gsd}
\NormalTok{y <-}\StringTok{ }\NormalTok{mgi }\OperatorTok{+}\StringTok{ }\NormalTok{fgi}

\NormalTok{R<-}\StringTok{ }\NormalTok{x}\OperatorTok{/}\NormalTok{y}
\KeywordTok{print}\NormalTok{(R)}
\end{Highlighting}
\end{Shaded}

\begin{verbatim}
## [1] 183333.3
\end{verbatim}

\section*{References}\label{references}
\addcontentsline{toc}{section}{References}

\hypertarget{refs}{}
\hypertarget{ref-Falconer1996}{}
Falconer, D. S., and Trudy F. C. Mackay. 1996. ``Response to
selection.'' In \emph{Introduction to Quantitative Genetics}, 4th ed.,
185--208. Essex: Addison Wesley Longman Limited.

\hypertarget{ref-VonRohr1998}{}
Von Rohr, Peter, N Künzi, J Krieter, and A Hofer. 1998.
``Wirtschaftliche Gewichte fiir Mastleistungs- und
Schlachtkörperqualitatsmerkmale beim Schwein.'' PhD thesis, ETH Zürich.


\end{document}
