\documentclass[]{article}
\usepackage{lmodern}
\usepackage{amssymb,amsmath}
\usepackage{ifxetex,ifluatex}
\usepackage{fixltx2e} % provides \textsubscript
\ifnum 0\ifxetex 1\fi\ifluatex 1\fi=0 % if pdftex
  \usepackage[T1]{fontenc}
  \usepackage[utf8]{inputenc}
\else % if luatex or xelatex
  \ifxetex
    \usepackage{mathspec}
  \else
    \usepackage{fontspec}
  \fi
  \defaultfontfeatures{Ligatures=TeX,Scale=MatchLowercase}
\fi
% use upquote if available, for straight quotes in verbatim environments
\IfFileExists{upquote.sty}{\usepackage{upquote}}{}
% use microtype if available
\IfFileExists{microtype.sty}{%
\usepackage{microtype}
\UseMicrotypeSet[protrusion]{basicmath} % disable protrusion for tt fonts
}{}
\usepackage[margin=1in]{geometry}
\usepackage{hyperref}
\hypersetup{unicode=true,
            pdftitle={Glossary},
            pdfauthor={Silvan},
            pdfborder={0 0 0},
            breaklinks=true}
\urlstyle{same}  % don't use monospace font for urls
\usepackage{color}
\usepackage{fancyvrb}
\newcommand{\VerbBar}{|}
\newcommand{\VERB}{\Verb[commandchars=\\\{\}]}
\DefineVerbatimEnvironment{Highlighting}{Verbatim}{commandchars=\\\{\}}
% Add ',fontsize=\small' for more characters per line
\usepackage{framed}
\definecolor{shadecolor}{RGB}{248,248,248}
\newenvironment{Shaded}{\begin{snugshade}}{\end{snugshade}}
\newcommand{\KeywordTok}[1]{\textcolor[rgb]{0.13,0.29,0.53}{\textbf{#1}}}
\newcommand{\DataTypeTok}[1]{\textcolor[rgb]{0.13,0.29,0.53}{#1}}
\newcommand{\DecValTok}[1]{\textcolor[rgb]{0.00,0.00,0.81}{#1}}
\newcommand{\BaseNTok}[1]{\textcolor[rgb]{0.00,0.00,0.81}{#1}}
\newcommand{\FloatTok}[1]{\textcolor[rgb]{0.00,0.00,0.81}{#1}}
\newcommand{\ConstantTok}[1]{\textcolor[rgb]{0.00,0.00,0.00}{#1}}
\newcommand{\CharTok}[1]{\textcolor[rgb]{0.31,0.60,0.02}{#1}}
\newcommand{\SpecialCharTok}[1]{\textcolor[rgb]{0.00,0.00,0.00}{#1}}
\newcommand{\StringTok}[1]{\textcolor[rgb]{0.31,0.60,0.02}{#1}}
\newcommand{\VerbatimStringTok}[1]{\textcolor[rgb]{0.31,0.60,0.02}{#1}}
\newcommand{\SpecialStringTok}[1]{\textcolor[rgb]{0.31,0.60,0.02}{#1}}
\newcommand{\ImportTok}[1]{#1}
\newcommand{\CommentTok}[1]{\textcolor[rgb]{0.56,0.35,0.01}{\textit{#1}}}
\newcommand{\DocumentationTok}[1]{\textcolor[rgb]{0.56,0.35,0.01}{\textbf{\textit{#1}}}}
\newcommand{\AnnotationTok}[1]{\textcolor[rgb]{0.56,0.35,0.01}{\textbf{\textit{#1}}}}
\newcommand{\CommentVarTok}[1]{\textcolor[rgb]{0.56,0.35,0.01}{\textbf{\textit{#1}}}}
\newcommand{\OtherTok}[1]{\textcolor[rgb]{0.56,0.35,0.01}{#1}}
\newcommand{\FunctionTok}[1]{\textcolor[rgb]{0.00,0.00,0.00}{#1}}
\newcommand{\VariableTok}[1]{\textcolor[rgb]{0.00,0.00,0.00}{#1}}
\newcommand{\ControlFlowTok}[1]{\textcolor[rgb]{0.13,0.29,0.53}{\textbf{#1}}}
\newcommand{\OperatorTok}[1]{\textcolor[rgb]{0.81,0.36,0.00}{\textbf{#1}}}
\newcommand{\BuiltInTok}[1]{#1}
\newcommand{\ExtensionTok}[1]{#1}
\newcommand{\PreprocessorTok}[1]{\textcolor[rgb]{0.56,0.35,0.01}{\textit{#1}}}
\newcommand{\AttributeTok}[1]{\textcolor[rgb]{0.77,0.63,0.00}{#1}}
\newcommand{\RegionMarkerTok}[1]{#1}
\newcommand{\InformationTok}[1]{\textcolor[rgb]{0.56,0.35,0.01}{\textbf{\textit{#1}}}}
\newcommand{\WarningTok}[1]{\textcolor[rgb]{0.56,0.35,0.01}{\textbf{\textit{#1}}}}
\newcommand{\AlertTok}[1]{\textcolor[rgb]{0.94,0.16,0.16}{#1}}
\newcommand{\ErrorTok}[1]{\textcolor[rgb]{0.64,0.00,0.00}{\textbf{#1}}}
\newcommand{\NormalTok}[1]{#1}
\usepackage{graphicx,grffile}
\makeatletter
\def\maxwidth{\ifdim\Gin@nat@width>\linewidth\linewidth\else\Gin@nat@width\fi}
\def\maxheight{\ifdim\Gin@nat@height>\textheight\textheight\else\Gin@nat@height\fi}
\makeatother
% Scale images if necessary, so that they will not overflow the page
% margins by default, and it is still possible to overwrite the defaults
% using explicit options in \includegraphics[width, height, ...]{}
\setkeys{Gin}{width=\maxwidth,height=\maxheight,keepaspectratio}
\IfFileExists{parskip.sty}{%
\usepackage{parskip}
}{% else
\setlength{\parindent}{0pt}
\setlength{\parskip}{6pt plus 2pt minus 1pt}
}
\setlength{\emergencystretch}{3em}  % prevent overfull lines
\providecommand{\tightlist}{%
  \setlength{\itemsep}{0pt}\setlength{\parskip}{0pt}}
\setcounter{secnumdepth}{0}
% Redefines (sub)paragraphs to behave more like sections
\ifx\paragraph\undefined\else
\let\oldparagraph\paragraph
\renewcommand{\paragraph}[1]{\oldparagraph{#1}\mbox{}}
\fi
\ifx\subparagraph\undefined\else
\let\oldsubparagraph\subparagraph
\renewcommand{\subparagraph}[1]{\oldsubparagraph{#1}\mbox{}}
\fi

%%% Use protect on footnotes to avoid problems with footnotes in titles
\let\rmarkdownfootnote\footnote%
\def\footnote{\protect\rmarkdownfootnote}

%%% Change title format to be more compact
\usepackage{titling}

% Create subtitle command for use in maketitle
\newcommand{\subtitle}[1]{
  \posttitle{
    \begin{center}\large#1\end{center}
    }
}

\setlength{\droptitle}{-2em}
  \title{Glossary}
  \pretitle{\vspace{\droptitle}\centering\huge}
  \posttitle{\par}
  \author{Silvan}
  \preauthor{\centering\large\emph}
  \postauthor{\par}
  \predate{\centering\large\emph}
  \postdate{\par}
  \date{6 8 2018}


\begin{document}
\maketitle

\subsection{R Markdown}\label{r-markdown}

This is an R Markdown document. Markdown is a simple formatting syntax
for authoring HTML, PDF, and MS Word documents. For more details on
using R Markdown see \url{http://rmarkdown.rstudio.com}.

When you click the \textbf{Knit} button a document will be generated
that includes both content as well as the output of any embedded R code
chunks within the document. You can embed an R code chunk like this:

\begin{Shaded}
\begin{Highlighting}[]
\KeywordTok{summary}\NormalTok{(cars)}
\end{Highlighting}
\end{Shaded}

\begin{verbatim}
##      speed           dist       
##  Min.   : 4.0   Min.   :  2.00  
##  1st Qu.:12.0   1st Qu.: 26.00  
##  Median :15.0   Median : 36.00  
##  Mean   :15.4   Mean   : 42.98  
##  3rd Qu.:19.0   3rd Qu.: 56.00  
##  Max.   :25.0   Max.   :120.00
\end{verbatim}

\subsection{Including Plots}\label{including-plots}

You can also embed plots, for example:

\includegraphics{Glossary_files/figure-latex/pressure-1.pdf}

Note that the \texttt{echo\ =\ FALSE} parameter was added to the code
chunk to prevent printing of the R code that generated the plot.

\subsection{R chunks}\label{r-chunks}

With ``alt'' + ``cmd'' + ``i'' you can create a r chunk:

\subsection{Lists}\label{lists}

There are two types of lists

\begin{enumerate}
\def\labelenumi{\arabic{enumi}.}
\tightlist
\item
  numbered lists
\item
  unnumbered lists
\end{enumerate}

\subsubsection{numbered lists}\label{numbered-lists}

Numbered lists start with a number and a dot.

\subsubsection{unnumbered lists}\label{unnumbered-lists}

Unnumbered lists start with a star.

\begin{itemize}
\tightlist
\item
  list item
\item
  list item
\end{itemize}

\subsection{formating}\label{formating}

Bold face with two \textbf{underscores} or \textbf{two stars}.

Italics with one \emph{underscore} or one \emph{star}.

Formulas and special characters in text with \(\sigma^2\)

Formulas and special characters in boxes with \[\sigma^2\]

\subsection{Citations}\label{citations}

This section explains how we do literature citations together with
Mendeley.

\begin{enumerate}
\def\labelenumi{\arabic{enumi}.}
\tightlist
\item
  First we have to export the *.bib-file for our papers by either

  \begin{itemize}
  \tightlist
  \item
    setting up automatic *.bib-generation in Mendeley-Desktop via the
    menu \texttt{Mendeley\ Desktop} \textgreater{} \texttt{Preferences}
    or by
  \item
    explicitly exporting the *.bib file from one paper by right click on
    the record and hit \texttt{Export}
  \end{itemize}
\item
  As the next step, we copy the *.bib file into the rmd-directory.
\item
  The paper can then be cited using the key of the bib-record from the
  *.bib-file.
\end{enumerate}

For more details see
\url{https://rmarkdown.rstudio.com/authoring_bibliographies_and_citations.html}

The following text is now citing our paper ({\textbf{???}}).

The next paper that we are citing is ({\textbf{???}})

\section{Glossary}\label{glossary}

\begin{itemize}
\item
  \textbf{Net growth} Mean gain in weight in kg per d. (
  \(carcass weight - 0.5 * birth weight) * 1000 / age\) (Kunz Sophie
  2018).
\item
  \textbf{Carcass weight} (SG) In the case of no legal definition,
  carcass weight should be defined as the hot weight of both half
  carcasses after being bled and eviscerated and after removal of skin,
  removal of external genitalia, the limbs at the carpus and tarsus,
  head, tail, kidneys and kidney fats and the udder (ICAR 2018). It is
  about what ICAR has as definition in the Swiss law (EDI 2013).
  Phenotypic carcass weight to produce a breeding value carcass weight.
  The weight data are collected in meat factories by a balance. It
  contains information about the enviromental and the genetic effects.
  Heritability of 0.22 for KV and 0.3 for MT show that breeding for
  carcass weight is possible. Breeding for a specific carcass weight is
  impossible. (Kunz S. and Strasser S. 2018). The higher the breeding
  value, the higher the potential carcass weight of the progeny?
\item
  \textbf{Early maturity} Either property of carcass fat coverage
  (optimum is class 3). Or property, how early an animal can be
  slaughtered when conditions (no price reductions per kg carcass
  weight) are met. Often expressed in the unit days. Should not be mixed
  up with sexual maturity. However corresponding to Andreas Landolt it
  may be linked to each other (Kunz Sophie 2018).
\item
  \textbf{Conformation} (F) Conformation is determined by the method
  CH-TAX. It tells you how much meat there is compared to the body
  weight (ABZ 2017).
\item
  \textbf{Weaning date} Date when the calf does not get any milk from
  its mother anymore.
\item
  \textbf{Tierverkehrsdatenbank} (TVD) There all animals are given a
  unique ID and informations about phenotypic performance, breeding
  values and relationship degrees are conserved. Identitas AG is owner
  of the data in Switzerland.
\item
  \textbf{Carcass category} Category of different types of carcasses
  depends on their age, sexual maturity and sex. MT = Bull ungeschaufelt
  \emph{Was sind Schaufeln?}, MA = Bull older, OB = Ox until 4
  Schaufeln, RG = young bull until 4 Schaufeln, RV = young bull/ young
  cow, cow until max 4 Schaufeln, VK = cow, JB = Young bull, KV = Calf
  (ABZ 2017)
\item
  \textbf{CH-TAX} Tool to categorize visually carcasses in different
  categories, it describes the trait carcass conformation. The price of
  the carcass depends on the category and the carcass weight. C = very
  well fleshed, H = well fleshed, T = medium fleshed, A = poorly
  fleshed, X = very poorly fleshed (ABZ 2017).
\item
  \textbf{Genetic correlation} How much are different traits genetically
  linked to each other. Also called shared heritability. If the term is
  0 the traits do not share any heritability with each other. If the
  term is 1 the traits share all the heritability with each other.
  Means: If one trait is higher in the next generation, the other will
  be higher by the same proportion. (proportional to each other). If the
  term is negative e.g. -1, the proportionality will be negative. If one
  trait is higher in the next generation the other one will be positive.
\item
  \textbf{Carcass fat coverage} (FET) There are 5 classes of fat
  coverages: 1 = non-covered, 2 = partially covered, 3 = evenly covered,
  4 = strongly covered, 5 = overfatty. The higher the breeding value of
  FET, the higher potentially the fat class of its offspring (Kunz S.
  and Strasser S. 2018).
\item
  \textbf{Estimation of variance components} Enables the calculation of
  heritabilities
  \url{https://qualitasag.ch/neu-im-juli-2018-update-zuchtwertschaetzung-schlachtmerkmale/}.
  Usually with a sample. You have multiple phenotypic values per farm,
  then you can calculate the variance (of the same animal?) accross
  different year over the same farm, you can calculate the variance of a
  estimator in the slaughterhouse per animal?.
\item
  \textbf{Qualität-DB} ?
\item
  \textbf{Pedigree} Pedigree of the animal(Rohr 2017). A tabulation of
  ids and names of an individual's ancestors. Pedigree information is
  used to establish genetic relationships among individuals to use in
  genetic evaluations (ICAR 2018).
\item
  \textbf{Mixed linear model} Trying to remove effects of the
  environment on the model by calculating it into the model. Farm is
  random effekt because there is not enough data to calculate its effect
  reliably (Kunz Sophie 2018).
\item
  \textbf{Additive genetic effect} The effect of the genetics of an
  animal that are inherited to its offspring. Not included is the
  dominance and epistasy effect (Rohr 2017).
\item
  \textbf{Animal model} Is a version of BLUP, where the phenotypic
  observation of an animal depends on the breeding value of the animal
  and the breeding value of the animal is dependent on the breeding
  values of all its relatives. Also the breeding value is corrected by
  the breeding value of its mating partners (Rohr 2017).
\item
  \textbf{Genetic groups} Groups of animals with unknown parents. Groups
  are formed according to age (year born), country of origin and/or
  breed composition (if more than one breed is included) (ICAR 2018).
  Example and purpose?
\item
  \textbf{Phantom parents} ?
\item
  \textbf{Basis animal} Animals in the pedigree that do not have parents
  according to the model?? (Bürgisser 2018). Or: One of the animals that
  create as their mean the population mean. E.G. In the german breeding
  valuation the animals from 2006-2010 are used to calculate the
  population mean for the breeding values 2014. They change each year in
  Germany (Vit, n.d.).
\item
  \textbf{Bankkalb} (KV) Watch carcass category. Why differentiate
  breeding values of KV and MT?
\item
  \textbf{Banktier} (MT) Watch carcass category
\item
  \textbf{Selection path} ?
\item
  \textbf{Variance components} ?
\item
  \textbf{Software VCE} ?
\item
  \textbf{Software Mix99} ?
\item
  \textbf{Residual} A residual is the vertical difference between a
  regression line and the observed value.
\item
  \textbf{Variance} (\(\sigma^2\)) It is the square mean deviation of
  the expected value. Here we use the empiric variance. It is calculated
  like in this example: We have a number row (vector) of:
\end{itemize}

\begin{Shaded}
\begin{Highlighting}[]
\NormalTok{vec_a <-}\StringTok{ }\KeywordTok{c}\NormalTok{(}\DecValTok{1}\NormalTok{,}\DecValTok{3}\NormalTok{,}\DecValTok{5}\NormalTok{,}\DecValTok{6}\NormalTok{,}\DecValTok{8}\NormalTok{,}\DecValTok{5}\NormalTok{)}
\end{Highlighting}
\end{Shaded}

We take the arithmetic mean out of the vector values:

\begin{Shaded}
\begin{Highlighting}[]
\NormalTok{mvec_a <-}\StringTok{ }\KeywordTok{mean}\NormalTok{(vec_a)}
\NormalTok{mvec_a}
\end{Highlighting}
\end{Shaded}

\begin{verbatim}
## [1] 4.666667
\end{verbatim}

We calculate the square differences (absolute) of the observations to
the mean/Erwartungswert.

\begin{Shaded}
\begin{Highlighting}[]
\NormalTok{sdvec_a <-}\StringTok{ }\NormalTok{(mvec_a}\OperatorTok{-}\NormalTok{vec_a)}\OperatorTok{^}\DecValTok{2}
\NormalTok{sdvec_a}
\end{Highlighting}
\end{Shaded}

\begin{verbatim}
## [1] 13.4444444  2.7777778  0.1111111  1.7777778 11.1111111  0.1111111
\end{verbatim}

We take the sum of the square differences and divide it by the degrees
of freedom which is the number of terms in the vector - 1.

\begin{Shaded}
\begin{Highlighting}[]
\KeywordTok{sum}\NormalTok{(sdvec_a)}\OperatorTok{/}\NormalTok{(}\KeywordTok{length}\NormalTok{(vec_a)}\OperatorTok{-}\DecValTok{1}\NormalTok{)}
\end{Highlighting}
\end{Shaded}

\begin{verbatim}
## [1] 5.866667
\end{verbatim}

Should be the same like

\begin{Shaded}
\begin{Highlighting}[]
\KeywordTok{var}\NormalTok{(vec_a)}
\end{Highlighting}
\end{Shaded}

\begin{verbatim}
## [1] 5.866667
\end{verbatim}

The advantage of using R-code-chunks to do the examples is that
computations are done automatically and do not lead to errors.
Furthermore, we do not have to do computations explicitly, but we can
use functions such as \texttt{sum} and \texttt{mean} which are readily
available.

\begin{itemize}
\tightlist
\item
  \textbf{Standard deviation} It is usually calculated as the square
  root of the variance. It is the mean absolute deviation to the mean of
  all observations or the Erwartungswert (when stetig) (Kalisch 2011).
  Like in the example of variance:
\end{itemize}

\begin{Shaded}
\begin{Highlighting}[]
\NormalTok{var_vec_a <-}\StringTok{ }\KeywordTok{var}\NormalTok{(vec_a)}
\NormalTok{stdev<-}\StringTok{ }\NormalTok{var_vec_a}\OperatorTok{^}\FloatTok{0.5}
\NormalTok{stdev}
\end{Highlighting}
\end{Shaded}

\begin{verbatim}
## [1] 2.42212
\end{verbatim}

\begin{itemize}
\item
  \textbf{Genetic variances} Phenotypic variances that can be explained
  by the genetics of the animal (by its pedigree) (Rohr 2017).
\item
  \_\_Farm*year variances\_\_ Phenotypic variances that can be explained
  by the environment (of which data is available like farm and year) of
  the animal. Different feed, managents, location (Hospenthal 2016).
\item
  \textbf{Heritability} (\(h^2\)) Selection success (mean of progeny)
  divided by selection differential (mean of selected parents). The
  higher, the more the trait is dependent on the genetics of the
  population (Rohr 2017).
\item
  \textbf{Selection differential} (S) The measure of the selection
  applied, which is the average superiority of the selected parents
  compared to the population mean ({\textbf{???}}).
\item
  \textbf{Intensity of selection} (i) The standardized selection
  differential, corrected by the phenotypic standard deviation of the
  values of a trait in the population. If you can assume a normal
  distribution you can derivate it from the proportion of the population
  selected as parents ({\textbf{???}}).
\item
  \textbf{Phenotypic standard deviation} (\(\sigma_{P}\)) You have
  values that have been measured for a certain trait. Of those values
  you can calculate the mean deviation of the mean of the values. This
  is the phenotypic standard deviation of a trait of a population
  ({\textbf{???}}).
\item
  \textbf{Response to selection} (R) \(R=ih^2\sigma_{P}\)
\item
  \textbf{Basis difference} ?
\item
  \textbf{Breeding valuation} (ZWS) Traditionally we use informations
  about phenotypic performances and degrees of relationships to estimate
  the breeding value. Now informations about the genetics of an animal
  can also be used (Rohr 2017). Also called genetic evaluation (ICAR
  2018).
\item
  \textbf{Training animal} Animal with own performance(Bürgisser 2018).
\item
  \textbf{Abbreviations Breeding valuation Mutterkuh Schweiz} ccc =
  Conformation Bankkälber, cca = Conformation Banktiere, cfc = Fat
  coverage Bankkälber, cfa = Fat coverage Banktiere, cwc = carcass
  weight Bankkälber, cwa = carcass weight Banktiere. (c1 = ?, c2 =
  conformation, f2 = fat coverage, w2 = carcassweight, c3 = calf, a3 =
  animal) (Kunz Sophie 2018).
\item
  \textbf{Abbreviations cow breeds} AN = Angus, AU = Aubrac, BV =
  Braunvieh, CH = Charolais, LM = Limousin, SM = Simmental (Kunz Sophie
  2018).
\item
  \textbf{Typify} Genomics not necessary yet.
\item
  \textbf{Data validation/selection} Making sure that the data is
  describing what we want to look at and deciding which data will be
  used for further analysis.
\item
  \textbf{Estimated breeding value} (EBV) EBVs are expressed in the
  units of measurement for each particular trait. They are shown as +
  ive or - ive differences between an individual animal's genetics
  difference and the genetic base to which the animal is compared. For
  example, a bull with an EBV of +50 kg for 600-Day Weight is estimated
  to have genetic merit 50 kg above the breed base of 0 kg. Since the
  breed base is set to an historical benchmark, the average EBVs of
  animals in each year drop over time as a result of genetic progress
  within the breed. The absolute value of any EBV is not critical, but
  rather the differences in EBVs between animals. Particular animals
  should be viewed as being ``above or below breed average'' for a
  particular trait.
  \url{http://abri.une.edu.au/online/pages/understanding_ebvs_char.htm}
\item
  \textbf{Breeding valuation carcass traits} Fat coverage, conformation,
  carcass weight (Kunz Sophie 2018).
\item
  \textbf{Breeding valuation early maturity} Only dependent on age and
  carcass weight?(Kunz Sophie 2018)
\item
  \textbf{Fat coverage} Trait describing the amount of fat on the meat
  of a carcass (Kunz Sophie 2018).
\item
  \textbf{Slaughterhouse} Where animals are slaughtered and
  processed.There are two categories of slaughterhouses in Switzerland:
  big and small. If there are more than 1500 slaughter units (per day?),
  it belongs to the category big, which has more reglementations to
  fulfill (Gresset, Python, and Réviron 2017).
\item
  \textbf{slaughter unit} I determined as 1 cow, 1 heifer/young bull, 2
  calves (Mail 2016).
\item
  \textbf{Degree of freedom} All possible values a random variable can
  express.
\item
  \textbf{ISET} ? Von Aktennotiz 16. Mai
\item
  \textbf{SHB} ? Von Aktennotiz 16. Mai
\item
  \textbf{FW} ? Von Aktennotiz 16. Mai
\item
  \textbf{IVF} ? Von Aktennotiz 16. Mai
\item
  \textbf{Nettozuwachs} (NZW) Wie genau definiert? durchschnittlicher
  Zuwachs in kg pro Tag seit Geburt?
\item
  \textbf{RH} ? Von Aktennotiz 16. Mai
\item
  \textbf{DAGE} Deviation (genetic difference) in age at slaughter of
  each animal relative to its contemporaries for constant carcass weight
  and constant subcutaneous carcass fatness. Animals with different
  subcutaneus carcass fatness then not comparable? Comes from concept
  residual feed intake Sitzung 23.8.17.
\item
  \textbf{Residual feed intake}
\item
  \textbf{Beef herdbook} (FLHB) The FLHB a part of Mutterkuh Schweiz.
  Its goal is to achieve genetic and economic improvements in suckle cow
  husbandry by collecting and analysing data for breeding (FLHB 2017).
\item
  \textbf{Beef performance control} (FLEK) FLEK is a phenotypic
  assessment of cattle. The traits assessed are growing performance of
  calves and reproduction performance of the dam (FLHB 2017).
\item
  \_\_Linear scoring (LB) LB is a phenotypic assessment of cattle. The
  traits assessed give information about the look of the dam or bull
  (FLHB 2017). The visual assessment of an animal for one or more
  morphological characteristics using a linear scale which represents
  the biological extremes in the population of animals under
  consideration (ICAR 2018).
\item
  \textbf{Population mean of breeding value} (\(\mu\)) Lies usually at
  100, when relative breeding values are used. It is calculated as the
  mean of the basis animals. If an animal has a breeding value over 100
  its offspring has the potential to be on average better in this trait
  than the average population (Kunz S. and Strasser S. 2018).
\item
  \textbf{QM Swiss meat} Meat that originates from animals kept on farms
  complying with ÖLN rules.
\item
  \textbf{Covariable} Connected to regressions (Stricker 2004). I still
  can not imagine it.
\item
  \textbf{Adjusted traits} Traits adjusted to facilitate more meaningful
  comparisons such as 365 day weights or eye muscle area adjusted to
  constant weight or age (ICAR 2018). Adjusted by covariables?
\item
  \textbf{Fixed / random effects} When a sample (? What kind) exhausts
  the population, the corresponding variable (?) is fixed; when the
  sample is a small (i.e., negligible) part of the population the
  corresponding variable is random (ICAR 2018). ?
\item
  \textbf{Best linear unbiased prediction} (BLUP) Is a solution of an
  equation system with many unknown variables. The solution is the
  breeding value of an animal. The equation system is fed by phenotypic
  performances of the observed animals and its relatives, the degree of
  relationship and environmental informations of the animals (Rohr
  2017).
\item
  \textbf{Multi-trait model} The breeding values of different traits are
  not calculated independently but are influence each other in using the
  knowledge about their shared heritability
  \url{https://qualitasag.ch/neu-im-juli-2018-update-zuchtwertschaetzung-schlachtmerkmale/}.
\item
  \textbf{Maternal genetic effect}
\item
  \textbf{Total breeding value} A calculated index of all breeding
  values to enable the simultaneous improvement of multiple breeding
  values.
  \url{https://qualitasag.ch/blick-hinter-die-kulissen-der-zuchtwertschaetzung-fuer-schlachtmerkmale-beim-rind/}
\item
  \textbf{Certainty} The more own or relative informations you have
  about a trait, the better you can compare the animals with each other
  and the more certain the breeding values are (Vit, n.d.)
\item
  \textbf{Phenotype} The set of observable charasteristics of an animal
  (ICAR 2018).
\item
  \textbf{Predictor trait} A trait that can be measured easily on an
  animal and which is highly correlated to a relevant production trait.
  An example is scrotal circumference in the case of male fertility
  (ICAR 2018).
\item
  \textbf{Progeny Test} he evaluation procedure of an animal based on
  the performance of its progeny (ICAR 2018).
\item
  \textbf{Steer} Castrated male (ICAR 2018).
\item
  \textbf{Base Population} A group of animals with unknown parents in
  genetic evaluations, whose EBV's are set to zero (ICAR 2018). Usually
  the base population consists of animals of the same
  breed.\url{https://qualitasag.ch/blick-hinter-die-kulissen-der-zuchtwertschaetzung-fuer-schlachtmerkmale-beim-rind/}
\item
  \textbf{Breed} Animals with a common origin and selection history.
  Animals within a breed have characteristics that distinguish them from
  other breeds or groups of animals within that same species (ICAR
  2018).
\item
  \textbf{Breeding farms} Farms which specialize in the production of
  animals to be used as sires and dams on other farms which rear animals
  for animal producing animal products (ICAR 2018).
\item
  \textbf{Calculated or Derived traits} Traits derived from recorded
  traits such as food conversion efficiency (ICAR 2018).
\item
  \textbf{Carcass composition} The proportions of a connective tissue,
  bone, muscle and fat in a carcass (ICAR 2018).
\item
  \textbf{Carcass grade} The relative level of a carcass for certain
  aspects, such as fatness, colour, maturity and class (e.g.~male,
  female, young, old) (ICAR 2018).
\item
  \textbf{Dressing percentage} Dressing percentage describes the percent
  ratio between carcass weight and the live weight taken immediately
  before slaughter (ICAR 2018).
\item
  \textbf{Empty weights} Animal weight when it consumes no feed or water
  for minimum 12 hours prior to weighing (ICAR 2018).
\item
  \textbf{Environmental effects} Effects which influence the performance
  of an animal which are not genetic in origin (ICAR 2018).
\item
  \textbf{E-U-R-O-P} EU carcass conformation system (ICAR 2018).
\item
  \textbf{Founder animal} The earliest known ancestor of an animal (ICAR
  2018).
\item
  \textbf{Generation interval} (\(L\)) Average age of parents when the
  offspring destined to replace them are born. It should be computed
  separately for male and female parents (ICAR 2018) {[}Falconer1996{]}.
\item
  \textbf{Heterosis} (Hybrid vigor) Amount by which the average
  performance for a trait in crossbred calves exceeds the average
  performance of the two or more purebreds that were mated in that
  particular cross (ICAR 2018).
\item
  \textbf{Indicator traits} Traits that do not have direct economic
  importance, but aid in the prediction of economically important traits
  (ICAR 2018).
\item
  \textbf{Invariant data} Data that is constant, such as an animals date
  of birth (ICAR 2018).
\item
  \textbf{Lean Meat} Meat with the connective tissue limits, but free of
  visible fat (ICAR 2018).
\item
  \textbf{Maintenance energy requirement} The amount of feed energy
  required per day by an animal to maintain its body weight and support
  necessary metabolic functions (ICAR 2018).
\item
  \textbf{Meat yield} The percentage of lean meat in the beef carcass as
  obtained by dissection (ICAR 2018).
\item
  \textbf{Trait category} Traits are categorized in two categories when
  we breed for them. One is called maximizing trait like carcass weight
  and conformation, where we breed only in one direction (more carcass
  weight). The other one is called optimizing trait like fat coverage
  where we try to move the population to an optimum value.
  \url{https://qualitasag.ch/blick-hinter-die-kulissen-der-zuchtwertschaetzung-fuer-schlachtmerkmale-beim-rind/}
\item
  \textbf{Post processing} Breeding values are getting based on their
  separate breed base e.g.~Simmental.
\item
  \textbf{Breed invariable} Tool to assess relative differences in
  traits across different breeds.
  \url{https://qualitasag.ch/blick-hinter-die-kulissen-der-zuchtwertschaetzung-fuer-schlachtmerkmale-beim-rind/}
\item
  \textbf{Genetic trend} Is calculated by the mean breeding values of
  following years.
  \url{https://qualitasag.ch/blick-hinter-die-kulissen-der-zuchtwertschaetzung-fuer-schlachtmerkmale-beim-rind/}
\item
  \textbf{Sire} Father animal.
\item
  \textbf{Dam} Mother animal.
\item
  \textbf{Heifer} Female which is had no calf yet.
\item
  \textbf{Young bull} Bull which is still in growth.
\item
  \textbf{Age at slaughter} Age of the animal that is slaughtered.
\item
  \textbf{Random regression modell}
\item
  \textbf{Economic weight} Provide indications as to how much genetic
  improvement in a specific trait would be worth paying for.
\item
  \textbf{String variable} Vector? (Stricker 2004)
\item
  \textbf{Expected value} The long-run average value of a random
  variable, that has many possibilities which value it can possess. In
  other words, each possible value the random variable can assume is
  multiplied by its probability of occurring, and the resulting products
  are summed to produce the expected value
  \url{https://en.wikipedia.org/wiki/Expected_value}. In a probability
  function the expected value is the on the long run most probable
  value. Usually when we have a sample, we take the arithmetic mean and
  the variance around the mean and say, that the sample is
  representative for infinite sampling points. The we can derive a
  function of probability through the points, which is defined by the
  mean and the variance of the sample.
\item
  \textbf{Index theory}
\item
  \textbf{Effect level}
\item
  \textbf{Ancestors} The progeny you descends from.
\item
  \textbf{Yearling weight}
\item
  \textbf{Genetic standard deviation}
\item
  \textbf{Index} Index values are reported as EBVs, in units of relative
  earning capacity (\$'s) for a given market. They reflect both the
  short-term profit generated by a sire through the sale of his progeny,
  and the longer-term profit generated by his daughters in a
  self-replacing cow herd. A selection index combines the EBVs with
  economic information (costs and returns) for specific market and
  production systems to rank animals based on relative profit values
  ({\textbf{???}}).
\item
  \textbf{Selection accuracy} {[}\%{]} is based on the amount of
  performance information available on the animal and its close
  relatives - particularly the number of progeny analysed. Accuracy is
  also based on the heritability of the trait and the genetic
  correlations with other recorded traits. Hence accuracy indicates the
  ``confidence level'' of the EBV. The higher the accuracy value the
  lower the likelihood of change in the animal's EBV as more information
  is analysed for that animal or its relatives ({\textbf{???}}).
\item
  \textbf{Mid-parent regression}
\item
  \textbf{Additive genetic variance} Square of the standard deviation of
  breeding values.
\item
  \textbf{Standard deviation of breeding values} (\$\sigma\_\{A\})
\end{itemize}

\[\sigma_{\bar{x_{1}} - \bar{x_{2}}}^2 = \frac {\sigma_{1}^2}{n_{1}} + \frac{\sigma_{2}^2}{n_{2}}\]
\[\sigma^2\]

\section*{References}\label{references}
\addcontentsline{toc}{section}{References}

\hypertarget{refs}{}
\hypertarget{ref-ABZ2017}{}
ABZ. 2017. \emph{Lebensmitteltechnologie ETH - Frischfleisch}. Edited by
ABZ / WDH. 4th ed. Spiez.

\hypertarget{ref-Burgisser2018}{}
Bürgisser, Monica. 2018. ``Protokoll FLHB-Kommission.'' In \emph{134.
Flhb-Kommissionssitzung}. Mutterkuh Schweiz.

\hypertarget{ref-EDI2013}{}
EDI. 2013. ``Schlachtgewichtsverordnung.''

\hypertarget{ref-FLHB2017}{}
FLHB. 2017. ``Das Fleischrinderherdebuch für eine erfolgreiche Zucht.''
Brugg: Mutterkuh Schweiz.

\hypertarget{ref-Gresset2017}{}
Gresset, Fabienne, Pascal Python, and Sophie Réviron. 2017.
``Wertschöpfungskette Rindfleisch.'' Lausanne: AGRIDEA.

\hypertarget{ref-Hospenthal2016}{}
Hospenthal, Andrea. 2016. ``Pilot-study to evaluate optimal ages at
slaughter in beef cattle.'' PhD thesis, ETH Zurich.

\hypertarget{ref-ICAR}{}
ICAR. 2018. ``Glossary for animal production sector.'' Accessed August
9.

\hypertarget{ref-Kalisch2011}{}
Kalisch, Markus. 2011. ``Statistik,'' no. September.

\hypertarget{ref-KunzS.2018}{}
Kunz S., and Strasser S. 2018. ``Merkblätter.'' Zug: Qualitas AG.

\hypertarget{ref-Kunz2018}{}
Kunz Sophie. 2018. ``Zuchtwertschätzung: Neue Schlachtmerkmale.'' Zug:
Qualitas AG.

\hypertarget{ref-VonRohr2017}{}
Rohr, Peter von. 2017. ``Züchtungslehre.'' Zug: Qualitas AG.

\hypertarget{ref-Vit}{}
Vit. n.d. ``ZWS Produktionsmerkmale - Relativzuchtwert Fleisch.'' vit.


\end{document}
